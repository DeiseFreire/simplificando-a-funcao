\documentclass[a4paper, portuguese, 12pt]{article} % Define a classe de documento (article para artigo)

% Pacotes
\usepackage[utf8]{inputenc}
\usepackage[T1]{fontenc}
\usepackage{amsmath}
\usepackage{pifont}
\usepackage{cancel}

% Início do documento
\begin{document}
\title{Simplificação de fração - funções $f$ e $g$} % Define o título
\author{Deise} % Define o autor

\maketitle % Imprime o título e o autor

\section*{Simplificação da Fração} % Define a seção
Sejam as funções $f(x) = x^{2} - 3$ e $g(x) = 2x - 1$, vamos simplificar a fração:

\[
f(x) = \frac{f(x) + g(x) - 2(x - 2)}{x^{2}}
\]
Seguindo os passos da simplificação:

\begin{align*}
    \frac{f(x) + g(x) - 2(x - 2))}{x^{2}} &= \frac{x^{2} - 3 + 2x - 1 - 2(x - 2)}{x^{2}} \\
    &= \frac{x^{2} - 3 + 2x - 1 - 2(x - 2)}{x^{2}} \\
    &= \frac{x^{2} - 3 + 2x - 1 - 2x + 4}{x^{2}} \\
    &= \frac{x^{2} + 2x - 2x - 1 - 3 + 4}{x^{2}} \\
    &= \frac{x^{2} + \cancel{2x} - \cancel{2x} - 4 + 4}{x^{2}} \\
    &= \frac{x^{2} - \cancel{4} + \cancel{4}}{x^{2}} \\
    &= \frac{x^{2}}{x^{2}} \\
    &= \frac{\cancel{x^{2}}}{\cancel{x^{2}}} \\
    &= \boxed{1}
\end{align*}

\end{document}